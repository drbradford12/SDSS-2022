% File SDSS2020_SampleExtendedAbstract.tex
\documentclass[10pt]{article}
\usepackage{sdss2020} % Uses Times Roman font (either newtx or times package)
\usepackage{url}
\usepackage{latexsym}
\usepackage{amsmath, amsthm, amsfonts}
\usepackage{algorithm, algorithmic}  
\usepackage{graphicx}

\title{Symposium on Data Science and Statistics (SDSS 2022) \\
Submission and Formatting Instructions for Extended Abstracts}

\author{
  Denise Bradford \\
  University of Nebraska - Lincoln \\
  Lincoln, Nebraska \\
  {\tt denise.bradford@huskers.unl.edu} \\\And
  Susan VanderPlas \\
  University of Nebraska - Lincoln \\
  Lincoln, Nebraska \\
  {\tt susan.vanderplas@unl.edu} \\}
  

\date{}

\begin{document}
\maketitle
\begin{abstract}
Interactive dashboards are the most common use cases for data visualization and context for exploratory data tools. In our paper, we will explore the specific scope of how dashboards are used in settings when the analysis is a novice to data. We will demonstrate the surrounding. Our framework will suggest a number of research directions to better support dashboard design, implementation and use for the every day analyst. 
\end{abstract}

{\bf Keywords:} Interactive Dashboards, Exploratory Data Analysis (EDA), Design Space(??)

\section{Introduction}
With the amount of publicly open-source data, a proliferation of visualization dashboards has increased in nearly every industry \cite{Fisher}. A dashboard in its fundamental form, a dashboard supports a way of presenting and making sense of complex data to better enable and support decision making. Stephen Few defines a dashboard as:
\begin{quotation}
\small A visual display of the most important information needed to achieve one or more objectives, consolidated and arranged on a single screen so the information can be monitored at a glance. \cite{Few}
\end{quotation}

\section{Methods}
Using an exploratory methodology, we derived a design space consisting of 3-4 visual and functional aspects of dashboards with statistical exploratory data analysis. see Section~\ref{info-on-sections}.

\subsection{Purpose}
The intended use of a dashboard for a novice user drives the choices in the visual and functional design allowances. The factors here capture the roles of each dashboard in the process of analysis and communication. We find that the purpose of a dashboard has been substantially expanded from the operational dashboard to capture decision-making at higher levels, and may not be supportive in decision-making.

{\bf Decision Support} (Strategic, Tactical, Operational): The decision support dimension reflects on the sorts of actionable decisions that dashboards are designed to support.

{\bf Communication and Learning:} We encountered several examples of dashboards that did not solicit decision-making on any temporal scale.

\subsection{Audience}
The visual and functional aspects of our dashboards typically reflect the intended audience, which has a wide-range of characteristics in their domain and visualization aspects of a dashboard and their experience with data. 

\section{Related Work}
From previous research has explained how visual analytics addresses the issue of information overload and enables analysts to transform raw data into salient information and knowledge 


\section{Data/Results}

We encourage the selective use of tables and/or figures as appropriate.

\subsection{Graphics and Tables}

Place figures and tables in the
paper near where they are first discussed, rather than at the end, if
possible.  It is acceptable that wide illustrations
may run across both columns.  The use of color
is allowed in extended abstract graphics.


\section{Discussion/Conclusions}

\subsection{Appendices and acknowledgements}

Please avoid the use of appendices and acknowledgements 
as separate sections.  Any pertinent material should be included in the
body of the extended abstract rather than in an appendix.  If 
acknowledgements are necessary then they should be made using footnotes.

If you wish to provide links to accompanying code, this is allowable because
SDSS encourages best practices for reproducible research.  However, all
conclusions and exposition relevant to the refereeing process should appear in
the body of the extended abstract; furthermore, we cannot guarantee that 
code will be checked or even consulted during the refereeing process.

\subsection{Refereeing Decisions}
It is expected that accepted papers will be presented in person at SDSS in Pittsburgh 
from June 4 to 6, 2020.
Accepted papers' authors will be notified in plenty of time to make registration and travel 
arrangements and for papers not accepted to submit an e-poster to the 
conference if they so choose.

\bibliographystyle{sdss2020} % Please do not change the bibliography style
\bibliography{SampleReferencesForExtendedAbstract}

\end{document}
