\documentclass[letterpaper,inpress]{jdsart}

\setcounter{page}{1}
\pubmonth{July}
\pubyear{2020}
\volume{xx}
\issue{xx}
\doi{0000}

\usepackage{color}
\usepackage{fancyvrb}
\newcommand{\VerbBar}{|}
\newcommand{\VERB}{\Verb[commandchars=\\\{\}]}
\DefineVerbatimEnvironment{Highlighting}{Verbatim}{commandchars=\\\{\}}
% Add ',fontsize=\small' for more characters per line
\usepackage{framed}
\definecolor{shadecolor}{RGB}{248,248,248}
\newenvironment{Shaded}{\begin{snugshade}}{\end{snugshade}}
\newcommand{\AlertTok}[1]{\textcolor[rgb]{0.94,0.16,0.16}{#1}}
\newcommand{\AnnotationTok}[1]{\textcolor[rgb]{0.56,0.35,0.01}{\textbf{\textit{#1}}}}
\newcommand{\AttributeTok}[1]{\textcolor[rgb]{0.77,0.63,0.00}{#1}}
\newcommand{\BaseNTok}[1]{\textcolor[rgb]{0.00,0.00,0.81}{#1}}
\newcommand{\BuiltInTok}[1]{#1}
\newcommand{\CharTok}[1]{\textcolor[rgb]{0.31,0.60,0.02}{#1}}
\newcommand{\CommentTok}[1]{\textcolor[rgb]{0.56,0.35,0.01}{\textit{#1}}}
\newcommand{\CommentVarTok}[1]{\textcolor[rgb]{0.56,0.35,0.01}{\textbf{\textit{#1}}}}
\newcommand{\ConstantTok}[1]{\textcolor[rgb]{0.00,0.00,0.00}{#1}}
\newcommand{\ControlFlowTok}[1]{\textcolor[rgb]{0.13,0.29,0.53}{\textbf{#1}}}
\newcommand{\DataTypeTok}[1]{\textcolor[rgb]{0.13,0.29,0.53}{#1}}
\newcommand{\DecValTok}[1]{\textcolor[rgb]{0.00,0.00,0.81}{#1}}
\newcommand{\DocumentationTok}[1]{\textcolor[rgb]{0.56,0.35,0.01}{\textbf{\textit{#1}}}}
\newcommand{\ErrorTok}[1]{\textcolor[rgb]{0.64,0.00,0.00}{\textbf{#1}}}
\newcommand{\ExtensionTok}[1]{#1}
\newcommand{\FloatTok}[1]{\textcolor[rgb]{0.00,0.00,0.81}{#1}}
\newcommand{\FunctionTok}[1]{\textcolor[rgb]{0.00,0.00,0.00}{#1}}
\newcommand{\ImportTok}[1]{#1}
\newcommand{\InformationTok}[1]{\textcolor[rgb]{0.56,0.35,0.01}{\textbf{\textit{#1}}}}
\newcommand{\KeywordTok}[1]{\textcolor[rgb]{0.13,0.29,0.53}{\textbf{#1}}}
\newcommand{\NormalTok}[1]{#1}
\newcommand{\OperatorTok}[1]{\textcolor[rgb]{0.81,0.36,0.00}{\textbf{#1}}}
\newcommand{\OtherTok}[1]{\textcolor[rgb]{0.56,0.35,0.01}{#1}}
\newcommand{\PreprocessorTok}[1]{\textcolor[rgb]{0.56,0.35,0.01}{\textit{#1}}}
\newcommand{\RegionMarkerTok}[1]{#1}
\newcommand{\SpecialCharTok}[1]{\textcolor[rgb]{0.00,0.00,0.00}{#1}}
\newcommand{\SpecialStringTok}[1]{\textcolor[rgb]{0.31,0.60,0.02}{#1}}
\newcommand{\StringTok}[1]{\textcolor[rgb]{0.31,0.60,0.02}{#1}}
\newcommand{\VariableTok}[1]{\textcolor[rgb]{0.00,0.00,0.00}{#1}}
\newcommand{\VerbatimStringTok}[1]{\textcolor[rgb]{0.31,0.60,0.02}{#1}}
\newcommand{\WarningTok}[1]{\textcolor[rgb]{0.56,0.35,0.01}{\textbf{\textit{#1}}}}

\usepackage[utf8]{inputenc}
\providecommand{\tightlist}{%
  \setlength{\itemsep}{0pt}\setlength{\parskip}{0pt}}


\usepackage{amsfonts,amsmath,amssymb,amsthm} \usepackage{booktabs} \usepackage{lipsum}

\begin{document}
\begin{frontmatter}

\title{My Paper Title}
\runtitle{My Paper Running Title}

\author[1,2]{
  \inits{A.}
  \fnms{Adam}
  \snm{Ahen}  \thanksref{1}  \ead{foo@bar.com}}
\author[2]{
  \inits{B.}
  \fnms{Brett}
  \snm{Berger}  \thanksref{2}}
\author[3]{
  \inits{C.}
  \fnms{Carl}
  \snm{Camp}}
\author[3]{
  \inits{D.}
  \fnms{David}
  \snm{Dodge}}

\thankstext[type=corresp,id=1]{Corresponding author}\thankstext[id=2]{Another footnote.}
\address[1]{Department A, 
  \institution{University of Achievement}, \cny{Country A}}
\address[2]{
  \institution{Institution B}, \cny{Country B}}
\address[3]{
  \institution{Institution C}, \cny{Country C}}

\begin{abstract}
This article presents a classification of disease severity for patients with cystic fibrosis (CF). CF is a genetic disease that dramatically decreases life expectancy and quality.
\end{abstract}

\begin{keywords}
\kwd{conditional distribution}\kwd{cystic fibrosis}\kwd{Kullback--Leibler divergence}\kwd{MCMC}\kwd{quantiles}.
\end{keywords}

\end{frontmatter}

\hypertarget{equations}{%
\section{Equations}\label{equations}}

Weibull distribution has the virtue of being a mathematically tractable model
and is versatile in terms of its applications in reliability, life data
analysis, actuarial science and others. Apart from being a potential model in
survival analysis and reliability engineering, it has a vast domain of other
applications.

Equations are always parts of sentences, so they need to have
appropriate punctuations. To evaluate the distribution of a normal
variable, one use
\begin{equation}
  \label{eq:cdf}
  \Pr(Z \le t) = \Phi\left(\frac{Z - \mu}{\sigma} \right),
\end{equation}
where \(Z\) follows a \(N(\mu, \sigma^2)\) distribution.
Equations can be referenced by \texttt{$\backslash$eqref}.
When \(\mu = 0\) and \(\sigma = 1\), the \(Z\) in Equation\textasciitilde{}\eqref{eq:cdf}
becomes a standard normal variable.

Multiline equations can be presented with the \texttt{align}
environment. For example,
\begin{align*}
  g_{\mu}(\phi) = 0,\\
  g_{\mu}(X) = 1.
\end{align*}

An equation that is not referenced should not be labeled. The starred
version of the \texttt{equation} and \texttt{align} are for this purpose.

\hypertarget{tables}{%
\section{Tables}\label{tables}}

We recommend \LaTeX~package \texttt{booktabs} for professional
looking tables. Its toprule and bottomrule are thicker than midrule.

A professional table contains no vertical lines.

\hypertarget{figures}{%
\section{Figures}\label{figures}}

Vector graphics do not lose clarity when being scaled. Make your
figure in pdf format when you first generate it and keep in mind its
sizes in the article to avoid over-scaling. Do not simply convert a
jpeg or png image to a pdf.

\hypertarget{code}{%
\section{Code}\label{code}}

The document class \texttt{jds} provides several commands to decorate

\begin{itemize}
\item inline code, such as \code{print("Hello world!")};
\item programming language, such as \proglang{R}, \proglang{Python}, and
  \proglang{C++};
\item software package, such as \pkg{stats}, \pkg{utils}.
\end{itemize}

\begin{Shaded}
\begin{Highlighting}[]
\DocumentationTok{\#\# Dobson (1990) Page 93: Randomized Controlled Trial :}
\NormalTok{counts }\OtherTok{\textless{}{-}} \FunctionTok{c}\NormalTok{(}\DecValTok{18}\NormalTok{, }\DecValTok{17}\NormalTok{, }\DecValTok{15}\NormalTok{, }\DecValTok{20}\NormalTok{, }\DecValTok{10}\NormalTok{, }\DecValTok{20}\NormalTok{, }\DecValTok{25}\NormalTok{, }\DecValTok{13}\NormalTok{, }\DecValTok{12}\NormalTok{)}
\NormalTok{outcome }\OtherTok{\textless{}{-}} \FunctionTok{gl}\NormalTok{(}\DecValTok{3}\NormalTok{, }\DecValTok{1}\NormalTok{, }\DecValTok{9}\NormalTok{)}
\NormalTok{treatment }\OtherTok{\textless{}{-}} \FunctionTok{gl}\NormalTok{(}\DecValTok{3}\NormalTok{, }\DecValTok{3}\NormalTok{)}
\NormalTok{glm.D93 }\OtherTok{\textless{}{-}} \FunctionTok{glm}\NormalTok{(counts }\SpecialCharTok{\textasciitilde{}}\NormalTok{ outcome }\SpecialCharTok{+}\NormalTok{ treatment, }\AttributeTok{family =} \FunctionTok{poisson}\NormalTok{())}
\FunctionTok{summary}\NormalTok{(glm.D93)}
\end{Highlighting}
\end{Shaded}

\begin{verbatim}
Call:
glm(formula = counts ~ outcome + treatment, family = poisson())

Deviance Residuals: 
       1         2         3         4         5         6         7         8         9  
-0.67125   0.96272  -0.16965  -0.21999  -0.95552   1.04939   0.84715  -0.09167  -0.96656  

Coefficients:
              Estimate Std. Error z value Pr(>|z|)    
(Intercept)  3.045e+00  1.709e-01  17.815   <2e-16 ***
outcome2    -4.543e-01  2.022e-01  -2.247   0.0246 *  
outcome3    -2.930e-01  1.927e-01  -1.520   0.1285    
treatment2   1.338e-15  2.000e-01   0.000   1.0000    
treatment3   1.421e-15  2.000e-01   0.000   1.0000    
---
Signif. codes:  0 '***' 0.001 '**' 0.01 '*' 0.05 '.' 0.1 ' ' 1

(Dispersion parameter for poisson family taken to be 1)

    Null deviance: 10.5814  on 8  degrees of freedom
Residual deviance:  5.1291  on 4  degrees of freedom
AIC: 56.761

Number of Fisher Scoring iterations: 4
\end{verbatim}

\hypertarget{guide-for-authors}{%
\section{Guide for Authors}\label{guide-for-authors}}

The following requirements must be followed as closely as possible. A
technically acceptable manuscript that fails to follow these requirements may be
returned for retyping, leading to delay in publication.We only accept
submissions in PDF format. The Latex file must be provided after the manuscript
is accepted.

\hypertarget{submission-of-papers}{%
\subsection{Submission of Papers}\label{submission-of-papers}}

Submission of a manuscript must be the original work of the author(s) and have
not been published elsewhere or under consideration for another publication, or
a substantially similar form in any language.

Authors are encouraged to recommend three to five individuals (including their
research fields, e-mail, phone numbers and addresses) who are qualified to serve
as referees for their paper.

\hypertarget{a-placeholder-section}{%
\section{A Placeholder Section}\label{a-placeholder-section}}

\lipsum[4-7]

\hypertarget{citing-references}{%
\section{Citing References}\label{citing-references}}

The citations are in the author-year format with the
\texttt{jds} bibstyle.

Citations can be in either text or parenthesis format style with
\texttt{$\backslash$citet} or \texttt{$\backslash$citep},
respectively. For example, \citet{KoenkerBassett1978} is a seminal
work on quantile regression; The Laplace distribution has applications
in many fields \citep{Kotz2001}.

\hypertarget{discussion}{%
\section{Discussion}\label{discussion}}

\lipsum[1-3]

\bibliography{jdsart-sample.bib}
\bibliographystyle{jds}


\end{document}
